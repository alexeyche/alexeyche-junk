\documentclass[a4paper,10pt]{article}
\usepackage[T2A]{fontenc}
\usepackage[utf8x]{inputenc}
\usepackage{ucs}
\usepackage{cmap}
\usepackage[english,russian]{babel}
\usepackage{amsmath}
\usepackage{color,graphicx}
\usepackage{indentfirst}

\title{SRM методичка}
\author{Чернышев Алексей}
\setlength{\parindent}{1cm}

\begin{document}
\section*{Спайковые НС}
\paragraph{Spike Responce Model.} Spike Responce Model (SRM) - наиболее популярная модель спайкового нейрона. SRM своей популярностью обязана простотой математической интерпретации - вся динамика нейрона описывается одним уравнением вида $u(t)$, которое описывает напряжение на мембране нейрона и, по сути, является решением дифференциального уравнения для моделей типа \textit{Integrate and fire}. Динамику моделей \textit{Integrate and fire} можно описать так: нейрон суммирует входные сигналы и по достижению определенного порога, производит спайк, после чего нейрон переходит в состояние рефракторности, находясь в котором, вероятность нового спайка крайне мала.\\
\indent Поведение описанное выше можно поэтапно собрать в одну формулу. Для начала формализуем вид входного сигнала на синапсах нейрона.
\end{document}
