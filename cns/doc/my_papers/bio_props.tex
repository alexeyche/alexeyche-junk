    \indent За последние сто лет биологические исследования накопили огромное количество детализированных знаний о нюансах функционирования мозга. В качестве структурной единицы центральной нервной системы, рассматривается клетка - нейрон, соединение которых в огромные формирования характеризует сложное  устройство нервных систем.\\
    \indent Не смотря на разнообразие типов нейронов, которые накопила нейронаука, можно выделить основные характеристики биологического нейрона:
    \begin{enumerate}
    \item нейрон, клетка имеющая опеределённый заряд, который поддерживается балансом между концентрацией ионов солей внутри клетки и снаружи
    \item динамику нейрона можно свести к стадии накопления потенциала и стадии выработки короткого импульса     (спайка)
    \item нейроны соединяются между собой синапсами, которые проводят спайки к другому нейрону (или дендриту нейрона)
    \item после выработки спайка из-за закрытия определённых ионных каналов, нейрон некоторое время не чувствителен к входным спайкам (состояние рефракторности)
    \item в нейроне, также как и во многих других клетках, существует феномен адаптации, т.е. динамика нейрона начинает угасать (привыкать) к одному и тому же стимулу
    \end{enumerate}
    Нейрон, в первую очередь, интересен своими возможностями по обработке информации и, до сих пор, важной является проблема выделения тех самых необходимых свойств биологического нейрона, моделирование которых поможет найти лучшее решения в сложных интеллектульных и перцептивных задачах.\\
    \indent В данном обзоре рассматриваются ряд нейронных моделей, каждая из которых несет в себе ту или иную степень биоподобности. Рассматриваются достоинства и недостатки таких моделей в контексте задач машинного обучения.
