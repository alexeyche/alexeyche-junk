\documentclass[a4paper,10pt]{article}
\usepackage[T2A]{fontenc}
\usepackage[utf8x]{inputenc}
\usepackage{ucs}
\usepackage{cmap}
\usepackage[english,russian]{babel}
\usepackage{amsmath}
\usepackage{color,graphicx}
\usepackage{indentfirst}
\usepackage{ucs} 
\usepackage[utf8x]{inputenc}

\title{Обзор спайковых сетей}
\author{Чернышев Алексей}
\setlength{\parindent}{1cm}
\def\la{\left\langle\rule{0pt}{3em}}
\def\ra{\right\rangle}
\newcommand{\HRule}{\rule{\linewidth}{0.5mm}}

\begin{document}
\begin{titlepage}
\begin{center}
% Title
\HRule \\[0.4cm]
{ \huge \bfseries Cпайковые нейронные сети \\[0.4cm] }

\HRule \\[1.5cm]
% Author and supervisor
\begin{minipage}{0.4\textwidth}
\begin{flushleft} \large
\emph{Автор:}\\
асп. Чернышев Алексей
\end{flushleft}
\end{minipage}
\begin{minipage}{0.4\textwidth}
\begin{flushright} \large
\emph{Научный руководитель:} \\
д.ф.-т.н. Карпенко А.П.
\end{flushright}
\end{minipage}

\vfill

% Bottom of the page
{\large Август 2014}

\end{center}
\end{titlepage}


\tableofcontents
\clearpage
\section{Нейронные модели}
	
	
\subsection{Нейрон МакКалока-Питтса}
   Первая модель нейрона, положившая начало нейроинформатике  - модель МакКаллока-Питтса. Эта модель прочно заложила фундамент теории нейронных сетей, и исследования новых свойств этой модели не   прекращаются по сей день.\\
   \indent Впервые, была реализована идея использовать нейрон, как вычислительный элемент. Раннее развитие данного направления в основном характеризуется попыткой рассмотреть нейроны, как элементы,        реализующие простейшие логические операции или преобразования. Впоследствии были созданы более сложные схемы, в которых данный нейрон соединяется в сети.\\
   \indent Ключевой особенностью данной модели является то, что нейрон представляется как взвешенный сумматор входных признаков.\\
   \indent Не смотря на ошеломляющий успех и широкое применения данной модели 


\end{document}
